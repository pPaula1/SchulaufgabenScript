\documentclass[12pt,a4paper]{article}
\usepackage[margin=2cm]{geometry}
\usepackage{graphicx}
\usepackage{tabularx}
\usepackage{array}
\usepackage{enumitem}
\usepackage{hyperref}
\usepackage{float}

\setlength{\parindent}{0pt}
\setlength{\parskip}{6pt}

% A simple checkbox
\newcommand{\checkbox}{\(\square\)}

% A line field
\newcommand{\linefield}[1]{\rule{#1}{0.4pt}}

% Task heading
\newcommand{\tasktitle}[2]{\vspace{6pt}\textbf{#1}\hfill /#2\par\vspace{4pt}}

\usepackage{amssymb}
\providecommand{\checkbox}{$\square$}

\begin{document}

\begin{tabular}{p{0.22\textwidth} p{0.76\textwidth}}
  \includegraphics[width=\linewidth]{C:/Users/TKast/OneDrive/Dokumente/GitHub/SchulaufgabenScript/logos/krs.png} &
  \renewcommand{\arraystretch}{1.3}
  \begin{tabular}{|p{0.40\textwidth}|p{0.25\textwidth}|p{0.25\textwidth}|}
    \hline
    \textbf{1. Schulaufgabe} & Fach: Mathematik & Datum: 2024-11-26 \\ \hline
    Name, Vorname: \linefield{7cm} & Klasse: \linefield{3cm} & Nr./Note: \linefield{3cm} \\ \hline
    \multicolumn{3}{|l|}{Zeitzuschlag: \linefield{12cm}} \\ \hline
    \multicolumn{3}{|l|}{Unterschrift Erziehungsberechtigte: \linefield{12cm}} \\ \hline
    \multicolumn{3}{|l|}{LRSt \checkbox \quad IRSt \checkbox \quad ILSt \checkbox} \\ \hline
  \end{tabular}
\end{tabular}

\vspace{8pt}

\tasktitle{Aufgabe 1: Fehlerbilder zu linearen Funktionen}{4.5}
Verbinde jede Zeichnung mit der dazu passenden Beschreibung des Fehlers. Fehlt eine passende Beschreibung, formuliere sie im letzten Kasten.

\begin{center}\includegraphics[width=\linewidth]{C:/Users/TKast/OneDrive/Dokumente/GitHub/SchulaufgabenScript/tasks/lineare_funktionen_fehlerbilder_01/images/layout.png}\end{center}


\par\textbf{Gesamtpunkte:} 4.5\par
\end{document}